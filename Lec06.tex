\pagebreak
\section{WEEK6-Normalization}
\begin{enumerate}
    \item What's Normalization? 
    \begin{itemize}
        \item Database normalization is the process of organizing the fields and tables of a relational database to minimize redundancy and dependency.
        \item Normalization involves dividing large tables into smaller(and less redundant) tables and defining relationships between them .
        \item The objective : Additions, deletions, and modifications of a field can be made in just one table and then propagated through the rest of the database using the defined relationships .
    \end{itemize}
    \item Normalization theory built around the concept of normal form.
    \item Normal form:
    \begin{itemize}
        \item A relation is said to be in a particular normal form if it satisfies a certain set of constraints.
        \item Each normal form has a set of constraints, if our follows these constraints, then our database is in that particular normal form.
        \item there are number of normal forms 1NF, 2NF, 3NF, BCNF ,4NF.
    \end{itemize}
    \item Functional dependence(FD):    A functional dependency is an association between two attributes of the same relation. And in this relation one of the attributes called determinant, and other attribute is called determined.A arrow B means A functional determines B.A functionally determines B means that B = F(A)\\
    Also A determines B == B dependent on A == A uniquely identifies B.
    \item Three type of anomalies in DBMS when it isn't normalized.
    \begin{itemize}
        \item Insertion: Add a new row into DB need have all data in all non-null column.
        \item Update: If we trying to update a item in a rows, we have to update all the raws that have same key(Otherwise the data will became inconsistent).
        \item Deletion: If we delete one type of data, we probably will delete more data than what we trying to delete. 
    \end{itemize}
    \item First Normal Form(1NF): Iff every entires of relation has at most a single value.
    \item Fully functional dependence: Y FFD on X, means Y FD on x and NOT FD on any proper subset of X.
    \item Second Normal Form(2NF): IFf it's in 1NF and every non-key attribute is fully dependent on the primary key.
    \item Transitive Dependency: A FD B B FD C then A FD C.
    \item Third Normal Form(3NF): Iff it's in 2NF and Every nonprim attribute is transitively dependt on the key.
    
\end{enumerate}
\section{Design Theory for Relational Databases}
\begin{enumerate}
    \item K is superkey for R if K fd all attribute of R
    \item K is key if K is a superkey, but not any proper subset of K is a superkey
    
    \item Inference rules\begin{itemize}
        \item Splitting rule: x 
        \item Transitive rule
        \item Trivial FDs
        \item Closure
    \end{itemize}
    
    \item BCNF iff for any a FD b, a is superkey
    \item how to get BCNF: check if there is a violation\begin{itemize}
        \item Check FD if LHS could reach all the attributes.
    \end{itemize}
    \item A relation with 2 attribute is automatically in BCNF.
    \item Minimal cover\begin{enumerate}
        \item Make RHS be single attributes
        \item Remove extraneous attribute(on LHS)
        \item Eliminate redundant functional dependencies
    \end{enumerate}
\end{enumerate}
\section{Desired properties of normalization}
\begin{enumerate}
    
    \item No redundancies and anomalies
    \item Recoverablitiy of information
    \item Preservation of original FD's
\end{enumerate}