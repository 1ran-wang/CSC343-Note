\pagebreak
\section{WEEK1-Database}

\begin{enumerate}
    \item What is Database: 
    \begin{itemize}
        \item Collection of information that exists over a period of time.
        \item The related information when placed in an organized form makes a database.
    \end{itemize}
    \item Operations on Databases:
    \begin{itemize}
        \item To add new information
        \item To view or retrieve the stored information.
        \item To modify or edit the existing information.
        \item To remove or delete the unwanted information
        \item arranging the information in a desired order etc.
        \item To sum up: Just a tool to manage all information.
    \end{itemize}
    \item Database and computers: There are two approaches to storing data in computers (File based / Database)
    
    \item Step 1 Database system evolved from file system:
    \begin{itemize}
        \item Allow you storage big amounts of data
        \item do not guarantee safety. (possibility of data lost)
        \item can not modify same file concurrently.
        \item No query language.
        \item hard to visit(need write programs to extract information)
    \end{itemize}
    \item step 2 Relational databases: key idea
    \begin{itemize}
        \item Think in tables way instead of bits.
        \item could show the relation between data.(in table)
        \item Queries could be expressed in language. (more efficiency)
    \end{itemize}
    \item  step 3 The dream system
    \begin{itemize}
        \item Allows to create new databases and specify their schema(logical structure of the data) in simple language.
        \item enables [data query] and [modification] still in simple language.
        \item Supports intelligent storage of very large amounts of data.
        \item allowed multiuser-access.
        \item recoverable.
    \end{itemize}
    \item Finally DBMS(Database Management System)
    \item a database is a collection of data managed by DBMS.
    
\end{enumerate}