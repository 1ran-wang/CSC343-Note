\pagebreak
\section{WEEK9-More on SQL 2}


\begin{enumerate}
    \item Constraints \begin{itemize}
        \item a data type to limit the kind of data.
        \item 2 type of constraints\begin{itemize}
            \item Refrential constraints: PRIMARY KEY, FOREIGN KEY.
            \item Value constraints: NULL UNIQUE CHECK
        \end{itemize}
        \item PRIMARY KEY:\begin{itemize}
            \item could be defined in two ways \begin{itemize}
                \item title CHAR(40) PRIMARY KEY.
                \item PRIMARY KEY (title, year)
            \end{itemize}
        \end{itemize}
        \item Foreign keys: \begin{itemize}
            \item In relational model tables are  related to each other throught common column.
            \item A column in one table is a Primary key of this table iff its value uniquely identifies each tuple. Such table is called a parent table.
            \item A culumn in another table that reference a primary key column in the parent table is know as a foreign key. this table is called a child table.
            \item Example a INT REFRENCES SOMETABLE(a)
            
            \item Self relationships: mngno INT NOT NULL REFRENCES empno
            \item Foreign key has to exist in parent table's primary key,NULL is acceptable unless we say no(add NOT NULL ).
            \item any deletion or update to Parent table that will cause Foreign key change, need handle in three way:\begin{enumerate}
                \item Default: Reject the modification
                \item Cascade: Deleted: deleted the value in child as well. Update: change values in Child.
                \item Set NULL Change the value in child into NULL
                \item Syntax: ON [DELETE,UPDATE] [SET NULL, CASCADE]
            \end{enumerate}
        \end{itemize}
        \item Value Constraints: \begin{itemize}
            \item Define if NULL is disallowed\\ A numeric NOT NULL\\ can be alerted with ALTER TABLE ABC ALTER COLUMN A DROP NOT NULL.
            \item Define if UNIQUE values are required\\ A NUMERIC UNIQUE \\ Nulls are still allowed to inserted multiple times. But combination with NULL and NOT NULL values are disallowed to be duplicated.
            \item Define CHECK for a set of values.\\ CONSTRAINT cname CHECK condition\\CHECK (condition)
        \end{itemize}
    \end{itemize}
    \item Views
    \begin{itemize}
        \item What is view: View is a virtual table, A relation definition of other tables.
        \item It takes a little space to store, because it only contains the definitions.
        \item The table that really stored values called base table.
        \item Syntax: CRATE VIEW name AS (Query)
        \item you can query it as a relation.
        \item Most of views are not modifiable. However there exist updateable views.
        \item insertion: If you want to update a new value into a updateable view, you have to  fill the other attributes out with NULL or
the default, and have a tuple that will yield the valid
insertion into ,and also it has to meet the condition in WHERE.
    \end{itemize}
    \item Triggers
    \begin{itemize}
        \item What's a trigger: A procedure that is fired when an INSERT DELETE UPDATE statement is issued.
        \item usage \begin{enumerate}
            \item To enforce complex integrity constraints
            \item To enforce complex business rules
            \item To customize complex security authorizations
        \end{enumerate} 
        \item Parts of a trigger:\begin{enumerate}
            \item Triggering event: INSERT DELETE or UPDATE
            \item Trigger Constraint : A Boolean expression that must be true for the trigger to fire. Using WHEN
            \item Trigger Action : SQL statements to be executed when trigger fired.
        \end{enumerate}
        \item Type of Triggers\begin{enumerate}
            \item Row level Triggers: Execute once for each row in a transaction. Will executed on all rows that affected by the command that enables the trigger.(When no rows affected, the trigger is not executed at all.)
            \item Statement level Triggers: Exectued only once for each statement.They are default types of trigger created via the CREATE TRIGGER command.
            \item Before Triggers:BEFORE trigger execute the trigger action before the triggering statement, It can eliminate useless statement.
            \item After Triggers: AFTER trigger executes the trigger action after the triggering statement is executed.
        \end{enumerate}
        \item Syntax of trigger: 
        \\ CREATE OR REPLACE TRIGGER PriceTrig\\AFTER UPDATE OF price ON Sells \\FOR EACH ROW WHEN(new.price > old.price + 1.00) \\BEGIN \\INSERT INTO RupoffBars\\VALUES(new.bar)\\END;
        \item Options\begin{enumerate}
            \item FOR EACH ROW: add this means this is a row level triggers
            \item AFTER/BEFORE ... UPDATE/DELETE/INSERT ON ....(UPDATE...OF...)
            \item ROW-level: INSERT: new DELETE: old UPDATE new/old can use them in a WHEN clause
            \item The Condition IF. THEN .. END IF
            \item The Action: BEGIN...END
        \end{enumerate}

    \end{itemize}
    \item Transaction management.
\end{enumerate}