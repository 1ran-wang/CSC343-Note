\pagebreak
\section{WEEK2-Relation Model}
\begin{enumerate}
\item[0.] Relational Integrity Constraints:\\
• Constraints are conditions that must hold on all
valid relation instances. There are three main types
of constraints:
    \begin{enumerate}
        \item Key constraints
        \item Entity integrity constraints
        \item Referential integrity constraints
    \end{enumerate}
    \item Key
    \begin{itemize}
        \item Super Key: Is a key that Unique in a relation.
        \item Key: Minimal super key.
        \item Candidate Key:\begin{itemize}
            \item Uniqueness: There are no two tuple in R has same candidate key value. 
            \item Inreducibility: There are no subkey has uniqueness.
        \end{itemize}
        
        \item Primary Key: It's candidate key that not null.
        \item Alternate Key: Candidate keys which are not chosen as the primary key are the Alternate Keys.
        \item Foreign Key: Refer to another table's primary key.(Otherwise its Null) Normally use to link two table together.
    \end{itemize}
    \item Entity Integrity :\\
• This rule states that in a relation, value of attribute of
a primary key cannot be null.
    \item Referential Integrity:\\
• It states that, if a foreign key exists in a relation,
either the foreign key value must match a primary key
value of some tuple in its home relation or must be
wholly null
\end{enumerate}