\pagebreak
\section{WEEK8-More on SQL}
 
\begin{enumerate}
    \item Aggregation operators\begin{enumerate}
        
    \item What's Aggregation operators: Operations that apply to entire columns and produce a single result.
    \item Covered in this course: Grouping Operator: AVG() COUNT() MIN() MAX() SUM().
    \item GROUP BY : \begin{itemize}
        \item followed SFW
        \item Result will be grouped by the values of attributes in GROUP BY. All the AGG will only apply to each group of data.
    \end{itemize}
    \item If using any Aggregation, all the elements must be Aggregated or An attribute on the GROUP BY list.
    \item HAVING :Same as WHERE but let you decide which groups to keep. (the attribute in HAVING must be either a grouping attribute or aggregated attribute
    \item NULL: NULL values are not contributes to a sum average or count and never be the min or max unless all values are NULL.
    \item which means, if a aggregation is apply to all NULL values then the result will be NULL, except COUNT will be zero.
    \item DISTINT: DISTINT will be proceed before the aggregation!.
    \end{enumerate}
    \item BAG and SET \begin{enumerate}
    \item UNION INTERSECT EXCEPT: have to use Bracket between two queries.
    \item force to SET: using DISTINT will force the result became a SET.(SELECT DISTINCT)
    \item  Rules for set:
    \begin{itemize}
        \item UNION: A UNION B A + B without any duplicates
        \item INTERSECTION: A INTERSECT B common elements in both A and B
        \item EXCEPT: A EXCEPT B elements that in  not in B
    \end{itemize}
    
    \item force to BAG: using ALL will force the result became a BAG.(SELECT (a) UNION ALL (b))
    \item Rules for bag:
    \begin{itemize}
        \item UNION: A UNION B All elements in A and in B WITH duplicates
        \item INTERSECTION: A INTERSECT B common elements with minimum number in each of subquery
        \item EXCEPT: A EXCEPT B A-B the number of elements will never goes to negative.
        \item to sum up: When you consider about bag, just think that elements with same value are difference.
    \end{itemize}
    \end{enumerate}
    \item Database Modifications:
    \begin{enumerate}
        \item Insert \begin{itemize}
            \item INSERT INTO Target relation) values;
            \item values could be a result of a query.
            \item Date insertion: \begin{itemize}
                \item Default format: 'YYYY-MM-DD'
                \item Format that also works: 'DD-Mon-YYYY'
                \item TO\_DATA(str,format)
            \end{itemize}
        \end{itemize}
        \item Deletetion:
            DELETE FROM (Relation) WHERE (Condition)
        \item Update: UPDATE (relation) SET a=a b=b c=c WHERE (Conditions).
    \end{enumerate}
    
\end{enumerate}
